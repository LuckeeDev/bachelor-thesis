\begin{abstract}
  La materia oscura è stata accettata negli ultimi decenni come possibile soluzione a diverse incongruenze fra previsioni e misure a livello cosmologico, alcune delle quali vengono brevemente presentate nell'introduzione di questo lavoro. Un grande filone di esperimenti di rivelazione indiretta sta attualmente cercando prove della sua esistenza osservando tramite telescopi i raggi cosmici provenienti da molteplici fonti astronomiche. Sebbene questo non abbia ancora portato a una scoperta confermata di segnali di materia oscura, ha permesso di porre limiti molto stringenti sulla cross section della sua annichilazione. Il confronto di questi limiti con la cross section prevista teoricamente per la materia oscura nel caso in cui essa sia una ``reliquia termica'' composta da WIMPs permette di porre limiti anche sulla massa della particella stessa. L'obiettivo della tesi è studiare come l'effetto quantistico non-relativistico detto \emph{Sommerfeld enhancement} possa impattare di diversi ordini di grandezza questi limiti. Con riferimento ai dati provenienti dalla collaborazione Fermi-LAT, si mostra che in assenza di \emph{Sommerfeld enhancement} il limite inferiore sulla massa è di circa \(80 \mathrm{GeV} \), mentre in presenza di \emph{Sommerfeld enhancement} diventa circa \(9.9 \mathrm{TeV} \). L'intero studio è condotto nell'approssimazione di Coulomb, per cui l'interazione è mediata da un mediatore non massivo. La validità di questa approssimazione viene verificata alla fine del lavoro.
\end{abstract}