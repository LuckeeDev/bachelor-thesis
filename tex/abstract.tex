\begin{abstract}
  La materia oscura è stata accettata negli ultimi decenni come possibile soluzione a diverse incongruenze fra previsioni e misure a livello cosmologico ed astrofisico, alcune delle quali vengono brevemente presentate nell'introduzione di questo lavoro. Un grande filone di esperimenti di rivelazione indiretta sta attualmente cercando prove della sua esistenza osservando tramite telescopi i raggi cosmici provenienti da molteplici fonti astronomiche. Sebbene non abbia ancora portato a una scoperta confermata di segnali di materia oscura, questo ha permesso di porre limiti molto stringenti sulla sezione d'urto della sua annichilazione. Il confronto di questi limiti con la sezione d'urto prevista teoricamente per la materia oscura nel caso in cui essa sia una ``reliquia termica'' permette di porre limiti sulla massa della particella stessa. L’obiettivo della tesi è l’approfondimento delle evidenze osservative di materia oscura, della sua rivelazione indiretta e dell’effetto quantistico non-relativistico detto \emph{Sommerfeld enhancement}, con attenzione al suo impatto sui suddetti limiti.
  In particolare, viene calcolato il \emph{Sommerfeld enhancement} in un modello specifico di materia oscura e studiato il suo impatto sull'interpretazione dei dati provenienti dalla collaborazione Fermi-LAT in merito ai raggi gamma da galassie nane sferoidali. Si ottiene che queste osservazioni implicano che, nel modello scelto, la materia oscura debba avere una massa superiore a circa \(10 \mathrm{TeV} \), mentre ignorare erroneamente questo effetto farebbe scendere tale limite a circa \(80 \mathrm{GeV} \).
  L'intero studio è condotto nell'approssimazione di Coulomb, per cui l'interazione è mediata da un mediatore non massivo. La validità di questa approssimazione viene verificata alla fine del lavoro.
\end{abstract}