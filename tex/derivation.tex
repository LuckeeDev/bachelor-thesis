\section*{Derivation}

\subsection*{Reduction to one-dimensional problem}

The two-body problem involving the supposed new mediator \(\phi \) and the two identical Dark Matter particles can be expressed through the following non-relativistic Schrödinger equation:
\begin{equation}
	-\frac{1}{2m} \nabla_1 ^2 \Psi(\vec{r}_1,\vec{r}_2)  - \frac{1}{2m} \nabla_2 ^2 \Psi(\vec{r}_1,\vec{r}_2)  + V(\vert \vec{r}_1 - \vec{r}_2 \vert ) \Psi(\vec{r}_1,\vec{r}_2) =E \Psi (\vec{r}_1,\vec{r}_2),
\end{equation}
where \(m\) is the mass of the two identical Dark Matter particles, \(\Psi\) is the wavefunction of the system, \(V\) is the potential describing the interaction mediated by \(\phi \) and \(E\) is the total energy of the system. The first step to solving this equation is to rewrite it as a one-body problem in the centre-of-mass frame through the following substitutions: \(\vec{r} \coloneqq  \vec{r}_1 - \vec{r}_2\), which represents the distance between the two particles, and \(\vec{R} \coloneqq (\vec{r}_1 + \vec{r}_2) / 2\), which is the position of the centre-of-mass. Since the two particles have the same mass, the reduced mass \(\mu \) is just \(\mu = m / 2\). One can easily show that the following relations hold:
\begin{equation}
	\nabla _1 = \frac{1}{2} \nabla _R + \nabla _r
	\qquad
	\nabla _2 = \frac{1}{2} \nabla _R - \nabla _r
\end{equation}
To do this, I will call \(x^i_1, x^i_2, X^i \) and \(x^i\) the coordinates of \(\vec{r}_1, \vec{r}_2, \vec{R} \) and \(\vec{R}\) respectively. By applying the chain rule,
\begin{equation}
	\frac{\partial }{\partial x^i_1} = \frac{\partial X^i}{\partial x^i_1} \frac{\partial }{\partial X^i} + \frac{\partial x^i}{\partial x^i_1} \frac{\partial }{\partial x^i} = \frac{1}{2} \frac{\partial }{\partial X^i} + \frac{\partial }{\partial x^i},
\end{equation}
and a similar result can be obtained for the second particle. With this result, the equation turns into a simpler form:
\begin{equation}
	\left[- \frac{1}{8\mu} \nabla _R^{2} - \frac{1}{2\mu } \nabla _r^2 + V(r)\right] \Psi (\vec{R},\vec{r})= E \Psi (\vec{R},\vec{r})
\end{equation}
This equation clearly suggests that the wavefunction can be separated into the product of a function describing the centre of mass and another one describing the system as seen from the centre-of-mass frame: \(\Psi (\vec{R},\vec{r}) = \psi_R \psi _r\). The problem can therefore be studied as two separate problems, and for the sake of this derivation we will only focus on the centre-of-mass description of it. The centre-of-mass Schrödinger equation is
\begin{equation}
	\left[- \frac{1}{2\mu }\nabla _r^2 + V(r)\right]\psi _r = E_{CM} \psi _r
\end{equation}
Energy is conserved, so we can write \(E_{CM} \) as the energy when the two particles are infinitely far apart, or else as the sum of their kinetic energies when the potential is substantially negligible. In the centre-of-mass frame, the two particles have the same speed \(v\) but opposite direction of movement, resulting in a relative velocity \(v_{rel} = 2v\). This means that \(E_{CM} = mv^2 = \frac{1}{2} \mu v_{rel}^2\), which is the energy of a particle of mass \(\mu \) moving at a speed of \(v_{rel} \). One can make the standard substitution \(k=\mu v_{rel} \) and obtain
\begin{equation}
	\left[- \frac{1}{2\mu } \nabla ^2 + V(r)\right] \psi _k (\vec{r}) = \frac{k^2}{2\mu } \psi _k (\vec{r}),
\end{equation}
which is the standard description of a particle moving in a potential. To simplify the notation, I decide to drop the subscript r on the \(\nabla _r ^2\) operator. In the following, I might omit to specify the dependence of the potential and of the wavefunction on the position.

\subsection*{Reduction to radial problem}

If there was no potential, a valid solution would be a simple plane wave traveling in the \(z\) direction:
\begin{equation}
	\psi _k^{(0)} = e^{ikz} 
\end{equation}
If we add a spherically symmetric interaction centred at the origin, initially modeled as a typical Yukawa potential \(V(r) =- \frac{\alpha}{r} e^{-m_{\phi } r }\), we expect the wavefunction to behave as the sum of a plane wave (the incident wave that gets scattered by the potential) and a spherical wave with intensity dependent on the angle formed with the \(z\) axis (the scattered wave) as \(r\) goes to infinity:
\begin{equation}\label{deriv:asymptotics}
	\psi _k \to e^{ikz} + f(\theta ) \frac{e^{ikr} }{r} \text{ as } r\to \infty 
\end{equation}
We know from standard quantum mechanics that solutions to the Schrödinger equation with a spherically symmetric potential can be separated as
\begin{equation}
	\psi_{klm} (r,\theta,\varphi) = R_{kl} (r) Y_l^m(\theta , \varphi ),
\end{equation}
where the radial functions \(R_{kl} \) satisfy the radial equation
\begin{equation}
	-\frac{1}{2\mu } \frac{1}{r^2} \frac{\mathrm{d}}{\mathrm{d}r}\left(r^2 \frac{\mathrm{d}}{\mathrm{d}r}  R_{kl}\right) + \left(V(r)+ \frac{1}{2\mu }\frac{l(l+1)}{r^2}\right)R_{kl} = \frac{k^2}{2\mu }R_{kl} 
\end{equation}
and the \(Y_m^l(\theta , \varphi )\) are the spherical harmonics. The incoming wave breaks the spherical symmetry and forbids any dependence on \(\varphi \), as there is nothing in the potential that could introduce it \cite{griffiths}. Since \(Y_l^m \propto e^{im \varphi} P_l(\cos \theta ) \), where \(P_l\) is the \(l\)-th Legendre polynomial, we are forced to set \(m=0\) and we get that \(\psi_{kl} \propto R_{kl} P_l (\cos \theta )\). This means that we can express our solution for a set \(k\) as the sum over all possible values of \(l\) of the base solutions, each with amplitude \(A_l\):
\begin{equation}\label{deriv:sum}
	\psi _k = \sum_{l} A_l R_{kl} (r) P_l (\cos \theta )
\end{equation}
In order to find the coefficients \(A_l\), we need to match the wavefunction expressed as a sum in \eqref{deriv:sum} to its asymptotic form in \eqref{deriv:asymptotics}.