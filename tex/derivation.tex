\chapter{Solving the equation}

\section{Reduction to one-dimensional problem}

The two-body problem involving the supposed new mediator \(\phi \) and the two identical Dark Matter particles can be expressed through the following non-relativistic Schrödinger equation:
\begin{equation}
	-\frac{1}{2m} \nabla_1 ^2 \Psi(\vec{r}_1,\vec{r}_2)  - \frac{1}{2m} \nabla_2 ^2 \Psi(\vec{r}_1,\vec{r}_2)  + V(\vert \vec{r}_1 - \vec{r}_2 \vert ) \Psi(\vec{r}_1,\vec{r}_2) =E \Psi (\vec{r}_1,\vec{r}_2),
\end{equation}
where \(m\) is the mass of the two identical Dark Matter particles, \(\Psi\) is the wavefunction of the system, \(V\) is the potential describing the interaction mediated by \(\phi \) and \(E\) is the total energy of the system. The first step to solving this equation is to rewrite it as a one-body problem in the centre-of-mass frame through the following substitutions: \(\vec{r} \coloneqq  \vec{r}_1 - \vec{r}_2\), which represents the distance between the two particles, and \(\vec{R} \coloneqq (\vec{r}_1 + \vec{r}_2) / 2\), which is the position of the centre-of-mass. Since the two particles have the same mass, the reduced mass \(\mu \) is just \(\mu = m / 2\). One can easily show that the following relations hold:
\begin{equation}
	\nabla _1 = \frac{1}{2} \nabla _R + \nabla _r
	\qquad
	\nabla _2 = \frac{1}{2} \nabla _R - \nabla _r
\end{equation}
To do this, I will call \(x^i_1, x^i_2, X^i \) and \(x^i\) the coordinates of \(\vec{r}_1, \vec{r}_2, \vec{R} \) and \(\vec{r}\) respectively. By applying the chain rule,
\begin{equation}
	\frac{\partial }{\partial x^i_1} = \frac{\partial X^i}{\partial x^i_1} \frac{\partial }{\partial X^i} + \frac{\partial x^i}{\partial x^i_1} \frac{\partial }{\partial x^i} = \frac{1}{2} \frac{\partial }{\partial X^i} + \frac{\partial }{\partial x^i},
\end{equation}
and a similar result can be obtained for the second particle. With this result, the equation turns into a simpler form:
\begin{equation}
	\left[- \frac{1}{8\mu} \nabla _R^{2} - \frac{1}{2\mu } \nabla _r^2 + V(r)\right] \Psi (\vec{R},\vec{r})= E \Psi (\vec{R},\vec{r})
\end{equation}
This equation clearly suggests that the wavefunction can be separated into the product of a function describing the centre of mass and another one describing the system as seen from the centre-of-mass frame: \(\Psi (\vec{R},\vec{r}) = \psi_R \psi _r\). The problem can therefore be studied as two separate problems, and for the sake of this derivation we will only focus on the centre-of-mass description of it. The centre-of-mass Schrödinger equation is
\begin{equation}
	\left[- \frac{1}{2\mu }\nabla _r^2 + V(r)\right]\psi _r = E_{CM} \psi _r
\end{equation}
Energy is conserved, so we can write \(E_{CM} \) as the energy when the two particles are infinitely far apart, or else as the sum of their kinetic energies when the potential is substantially negligible. In the centre-of-mass frame, the two particles have the same speed \(v\) but opposite direction of movement, resulting in a relative velocity \(v_{rel} = 2v\). This means that \(E_{CM} = mv^2 = \frac{1}{2} \mu v_{rel}^2\), which is the energy of a particle of mass \(\mu \) moving at a speed of \(v_{rel} \). One can make the standard substitution \(k=\mu v_{rel} \) and obtain
\begin{equation}
	\left[- \frac{1}{2\mu } \nabla ^2 + V(r)\right] \psi _k (\vec{r}) = \frac{k^2}{2\mu } \psi _k (\vec{r}),
\end{equation}
which is the standard description of a particle moving in a potential. To simplify the notation, I decide to drop the subscript r on the \(\nabla _r ^2\) operator. In the following, I might omit to specify the dependence of the potential and of the wavefunction on the position.

\section{Reduction to radial problem}

If there was no potential, a valid solution would be a simple plane wave traveling in the \(z\) direction:
\begin{equation}
	\psi _k^{(0)} = e^{ikz} 
\end{equation}
If we add a spherically symmetric interaction centred at the origin, initially modeled as a typical Yukawa potential \(V(r) =- \frac{\alpha}{r} e^{-m_{\phi } r }\), we expect the wavefunction to behave as the sum of a plane wave (the incident wave that gets scattered by the potential) and a spherical wave with intensity dependent on the angle formed with the \(z\) axis (the scattered wave) as \(r\) goes to infinity:
\begin{equation}\label{deriv:asymptotic}
	\psi _k \to e^{ikz} + f(\theta ) \frac{e^{ikr} }{r} \text{ as } r\to \infty 
\end{equation}
We know from standard quantum mechanics that solutions to the Schrödinger equation with a spherically symmetric potential can be separated as
\begin{equation}
	\psi_{klm} (r,\theta,\varphi) = R_{kl} (r) Y_l^m(\theta , \varphi ),
\end{equation}
where the radial functions \(R_{kl} \) satisfy the radial equation
\begin{equation}\label{deriv:radial}
	-\frac{1}{2\mu } \frac{1}{r^2} \frac{\mathrm{d}}{\mathrm{d}r}\left(r^2 \frac{\mathrm{d}}{\mathrm{d}r}  R_{kl}\right) + \left(V(r)+ \frac{1}{2\mu }\frac{l(l+1)}{r^2}\right)R_{kl} = \frac{k^2}{2\mu }R_{kl} 
\end{equation}
and the \(Y_m^l(\theta , \varphi )\) are the spherical harmonics. The incoming wave breaks the spherical symmetry and forbids any dependence on \(\varphi \), as there is nothing in the potential that could introduce it \cite{griffiths}. Since \(Y_l^m \propto e^{im \varphi} P_l(\cos \theta ) \), where \(P_l\) is the \(l\)-th Legendre polynomial, we are forced to set \(m=0\) and we get that \(\psi_{kl} \propto R_{kl}(r) P_l (\cos \theta )\). This means that we can express our solution for a set \(k\) as the sum over all possible values of \(l\) of the base solutions, each with amplitude \(A_l\):
\begin{equation}\label{deriv:sum}
	\psi _k = \sum_{l} A_l R_{kl} (r) P_l (\cos \theta )
\end{equation}

\subsection{Asymptotic behavior of the radial functions}
In order to find the coefficients \(A_l\), we need to match the wavefunction expressed as a sum in Eq. \eqref{deriv:sum} to its asymptotic form in Eq. \eqref{deriv:asymptotic}. We now make the substitution \(R_{kl} \eqqcolon \chi_{kl} / r\) with the condition that \(\chi_{kl} (0)=0\) to prevent \(R_{kl} \) from blowing up at the origin, and Eq. \eqref{deriv:radial} reduces to
\begin{equation}\label{deriv:radial_chi}
	\left[- \frac{1}{2\mu }\frac{\mathrm{d}^2}{\mathrm{d}r^2} + \frac{l(l+1)}{2\mu r^2}+ V(r)\right]\chi_{kl} = \frac{k^2}{2\mu }\chi_{kl},
\end{equation}
where, for large \(r\), we can neglect the centrifugal term and the potential, leading to
\begin{equation}\label{deriv:approx_radial}
	\left[ \frac{\mathrm{d}^2}{\mathrm{d}r^2} + k^2  \right] \chi_{kl}  \simeq 0 \text{ as } r \to \infty 
\end{equation}
The general solution to Eq. \eqref{deriv:approx_radial} is
\begin{equation}
	\chi_{kl} (r) \simeq A e^{ikr} + B e^{-ikr}
\end{equation}
If we see Eq. \eqref{deriv:radial_chi} as a one-dimensional problem with \(V(r < 0) = +\infty \), we can interpret the second term of the solution as a left-traveling wave with amplitude \(B\) being reflected by the potential wall at \(r=0\) into a right-traveling wave with amplitude \(A\). Since no transmitted wave can exist, we have that \(\vert A \vert = \vert B \vert \), \(A\eqqcolon \vert A \vert e^{i\varphi_A} \) and \(B \eqqcolon  \vert A \vert e^{i\varphi_B}\). With the following substitutions
\begin{equation}
	\vert A \vert e^{i \varphi_A} \eqqcolon \frac{C}{2i} e^{-i \beta_l}
	\qquad
	\vert A \vert e^{i \varphi _B} \eqqcolon - \frac{C}{2i} e^{i \beta _l},
\end{equation}
we get
\begin{align}
	\chi_{kl}(r) &\underset{r \to \infty }{\simeq} \vert A \vert \left[ e^{ikr} e^{i \varphi_A}+ e^{-ikr} e^{i \varphi_B} \right] =\\
	&= C \sin (kr - \beta _l)
\end{align}
It can be shown \cite{cohen-tannoudji} that for an identically null potential \(\beta _l = \frac{l \pi }{2}\), therefore it's convenient to take this as a reference point and write
\begin{equation}\label{deriv:chi_asymptotic}
	\chi_{kl} (r) \underset{r \to \infty }{\simeq} \sin\left(kr- l \frac{\pi}{2} + \delta _l\right),
\end{equation}
where \(\delta _l\) is the phase shift caused by the introduction of the potential and where we have chosen \(A\) and \(B\) such that \(C=1\). The entire theory of scattering reduces to calculating the phase shift, but for the sake of this problem we will not need to do it.

\subsection{Asymptotic behavior of the solution}
The exponential \(e^{ikz} \), which is the solution in the absence of a potential, can be expanded in terms of spherical Bessel functions and Legendre polynomials as follows \cite{cohen-tannoudji}:
\begin{equation}
	e^{ikz} = \sum_{l} i^l (2l+1) j_l(kr)P_l(\cos \theta )
\end{equation}
The spherical Bessel functions behave as \(j_l(kr) \simeq \frac{1}{kr} \sin \left(kr - l \pi /2\right)\) as \(r \to \infty \), therefore the asymptotic behavior of the exponential for large \(r\) is
\begin{equation}\label{deriv:exp_asymptotic}
	e^{ikz} \to  \frac{1}{2ikr} \sum_{l} (2l+1) P_l (\cos \theta ) \left[ e^{ikr}  - (-1)^l e^{-ikr} \right]
\end{equation}
We now need to understand how the solution behaves after introducing a potential. Since angular momentum is conserved, each term of the series scatters independently with no change in amplitude, but only in phase. If we think about the equivalent one-dimensional problem once again, we can see the second term inside the square brackets in Eq. \eqref{deriv:exp_asymptotic} as an incoming left-traveling wave that scatters off of the infinite potential wall at \(r=0\) and is reflected into a right-traveling wave picking up a phase shift of \(2\delta _l\) \cite{griffiths}. The resulting asymptotic form of the solution is the following:
\begin{equation}\label{deriv:sol_asymptotic}
	\psi_k \to \frac{1}{2ikr}\sum_{l} (2l+1)P_l(\cos \theta )\left[e^{i(kr+2\delta _l)}- (-1)^l e^{-ikr}  \right]
\end{equation}

\subsection{Determining the expansion coefficients}
We are now ready to determine the expansion coefficients \(A_l\). In order to do this, let us equate the asymptotic form of the solution in Eq. \eqref{deriv:sol_asymptotic} to the asymptotic form of Eq. \eqref{deriv:sum} by plugging in the result from Eq. \eqref{deriv:chi_asymptotic}:
\begin{equation}
	\sum_{l} \frac{(2l+1)}{2ikr} P_l(\cos \theta ) \left[e^{i(kr + 2 \delta _l)} - (-1)^{l} e^{-ikr} \right] \overset{!}{=} \frac{1}{r} \sum_{l} A_l P_l(\cos \theta ) \sin (kr-l \frac{\pi}{2} + \delta _l)
\end{equation}
Corresponding terms in the series must be equal to each other, therefore
\begin{gather}
	\frac{(2l+1)}{k}\left[ e^{i(kr+2\delta _l)}-(-1)^l e^{-ikr} \right]
	=
	A_l \left[e^{i(kr + \delta _l)} e^{- il \frac{\pi }{2} } - e^{-i(kr + \delta _l)}e^{il \frac{\pi }{2}} \right]\\
	A_l = \frac{2l+1}{k} e^{i \delta _l} i^l\\
	\implies 
	\psi _k = \frac{1}{k} \sum_{l} (2l+1)i^l e^{i \delta _l} P_l (\cos \theta ) R_{kl} (r) \label{deriv:expansion}
\end{gather}

\subsection{Sommerfeld enhancement factor}
The Sommerfeld enhancement factor, previously defined as \(S_k = \vert \psi _k(0) \vert ^2 \), can now be easily calculated from the result in Eq. \eqref{deriv:expansion}. In order to do this, we are going to study the behavior of the radial functions \(R_{kl} \) around the origin by plugging the trial solution \(\chi_{kl} = r^{p+1} \) with the constraint that \(p\geq 0\) into Eq. \eqref{deriv:radial_chi} approximated around the origin. Indeed, if the potential doesn't blow up faster than \(1 / r\) as \(r \to 0\), we can neglect it when compared to the kinetic terms and get:
\begin{gather}
	\frac{\mathrm{d}^2}{\mathrm{d}r^2} r^{p+1} \simeq \frac{l(l+1)}{r^2}r^{p+1} \text{ as } r \to 0 \\
	\implies p(p+1)=l(l+1),
\end{gather}
which has solutions \(p=l\) and \(p=-l-1\). The second solution is not physically acceptable as it would make \(R_{kl} \) blow up at the origin, therefore \(R_{kl} \simeq r^l \) as \(r \to 0\). This means that the only term in the series that doesn't vanish at \(r=0\) is the one with \(l=0\), and that
\begin{equation}
	S_k=\vert \psi _k (0) \vert ^2 
	= \left\vert \frac{R_{k,l=0}(0)}{k} \right\vert ^2 
	= \left\vert \frac{1}{k}\lim_{r \to 0} \frac{\chi_{k,l=0}}{r} \right\vert ^2
	= \left\vert \frac{1}{k}\frac{\mathrm{d}\chi_{k,l=0}}{\mathrm{d}r} (0) \right\vert ^2
\end{equation}
We are now only interested in finding \(\chi_{k,l=0}\), which I will now simply rename to \(\chi \), a task that can be accomplished by solving the radial equation \eqref{deriv:radial_chi} in the specific case of \(l=0\):
\begin{equation}\label{deriv:goal_equation}
	-\frac{1}{2\mu } \frac{\mathrm{d}^2 \chi}{\mathrm{d}r^2} + V(r) \chi = \frac{k^2}{2\mu } \chi,
\end{equation}
with the boundary conditions that \(\chi (0) \to  0\) as \(r \to 0\) and that \(\chi (r) \to \sin (kr + \delta)\) as \(r \to \infty \) as imposed by Eq. \eqref{deriv:chi_asymptotic}.

\section{Radial equation solution}
In the Coulomb approximation, \(m_{\phi }=0 \) and the Yukawa potential turns into a simple Coulomb potential:
\begin{equation}\label{deriv:potential}
	V(r)=-\frac{\alpha}{r}
\end{equation}
In order to simplify Eq. \eqref{deriv:goal_equation} with the potential in Eq. \eqref{deriv:potential}, I'm going to perform the following substitution: \(r \eqqcolon \frac{x}{2\alpha \mu } \), which leads to
\begin{equation}
	\frac{\mathrm{d}^2}{\mathrm{d}r^2} = 4 \alpha ^2 \mu ^2 \frac{\mathrm{d}^2}{\mathrm{d}x^2} \text{ and } V(x) = - \frac{2\alpha ^2\mu}{x} 
\end{equation}
Eq. \eqref{deriv:goal_equation} now assumes the simpler form
\begin{equation}\label{deriv:final_equation}
	- \chi^{\prime\prime} - \frac{1}{x} \chi = \varepsilon ^2 \chi,
\end{equation}
where the prime sign denotes derivation with respect to \(x\) and where I have defined \(\varepsilon \coloneqq \frac{v_{rel} }{2 \alpha }\). With these new variables, the Sommerfeld enhancement factor assumes the following form:
\begin{equation}
	S_{\varepsilon } = \left\vert \frac{1}{\varepsilon } \frac{\mathrm{d}\chi }{\mathrm{d}x} (0) \right\vert^2
\end{equation}

\subsection{Solution ansatz}
In order to guess what the solution to this equation might look like, I'm going to study the asymptotic behavior of the equation at the origin and at infinity. At infinity, the \(1 / x\) term can be neglected compared to the \(\varepsilon ^2\) term and the equation is just
\begin{equation}
	\chi ^{\prime\prime} \simeq - \varepsilon ^2 \chi \text{ as } x \to \infty 
\end{equation}
Its solution is an exponential, \(\chi (x) \simeq e^{i \varepsilon x}\) as \(x \to \infty \). On the other hand, as \(x\) approaches \(0\), the potential term blows up and the constant \(\varepsilon ^2\) term becomes negligible, giving the following:
\begin{equation}
	\chi ^{\prime\prime} \simeq - \frac{1}{x} \chi \text{ as } x \to 0
\end{equation}
The solution to this second equation is not as straightforward as the first one, but it can be solved by substituting \(t^2\coloneqq x\) and expressing it as a Bessel equation. This leads to a solution of the form
% TODO: should I do this explicitly? Or should I mention that I used Mathematica?
\begin{equation}
	\chi (x) \underset{x \to 0}{\simeq} C_1 \sqrt{x} J_1 (2 \sqrt{x} ) + C_2 \sqrt{x} Y_1(2 \sqrt{x} ),
\end{equation}
where \(J_1\) and \(Y_1\) are the Bessel functions of the first and second order, respectively. However, \(\sqrt{x} Y_1 (2\sqrt{x} ) \) blows up as \(x\) approaches \(0\), and so we must set \(C_2 = 0\). Moreover, the Taylor expansion for the first order Bessel function is simply \(J_1 (2 \sqrt{x} ) \simeq \sqrt{x} \) as \(x \to 0\), therefore
\begin{equation}
	\chi (x) \underset{x \to 0}{\simeq} C_1x
\end{equation}
I am therefore going to try the following ansatz for the solution to this equation: \(\chi (x) \eqqcolon x e^{i\varepsilon x}v(x)\). It's important to note that this is not an approximation, because \(v(x)\) will take care of the interpolation between the two asymptotic behaviors of the function.

\subsection{Confluent hypergeometric equation}
Taking the second derivative of \(\chi \) gives
\begin{equation}
	\chi ^{\prime\prime} = e^{i\varepsilon x} \left[ (2i\varepsilon - x \varepsilon ^2) + (2 + 2 i \varepsilon x ) v^{\prime} + x v^{\prime\prime}   \right] 
\end{equation}
When plugging this into Eq. \eqref{deriv:final_equation}, we get
\begin{equation}
	x v^{\prime\prime} + (2 + 2 x i \varepsilon ) v^{\prime} + (1 + 2i \varepsilon ) v = 0,
\end{equation}
which can be brought into the standard confluent hypergeometric equation form by performing the substitution \(z \coloneqq - 2i \varepsilon x\). This gives
\begin{equation}
	\frac{\mathrm{d}}{\mathrm{d}x} = - 2 i \varepsilon \frac{\mathrm{d}}{\mathrm{d}z} 
	\text{ and } 
	\frac{\mathrm{d}^2}{\mathrm{d}x^2} = - 4 \varepsilon ^2 \frac{\mathrm{d}^2}{\mathrm{d}z^2} 
\end{equation}
and the equation turns into
\begin{equation}\label{deriv:hypergeometric_equation}
	z v^{\prime\prime} + (2-z) v^{\prime} - \left( 1- \frac{i}{2\varepsilon } \right) v = 0,
\end{equation}
which is the confluent hypergeometric equation with \(b=2\) and \(a=1- \frac{i}{2\varepsilon }\).
The standard solution to Eq. \eqref{deriv:hypergeometric_equation} is a linear combination of Kummer's confluent hypergeometric function \(M(a,b,z)\) and Tricomi's confluent hypergeometric function \(U(a,b,z)\), however \(U(a,b,z)\) has a singularity at \(z=0\) and therefore must be excluded from our solution. The solution to Eq. \eqref{deriv:final_equation} is
\begin{equation}
	\chi (x) = Cxe^{i\varepsilon x} M\left(1-\frac{i}{2\varepsilon }, 2, -2i\varepsilon x\right)
\end{equation}
All that's left to do is to determine the normalization constant \(C\) that satisfies the boundary condition set in Eq. \eqref{deriv:chi_asymptotic}. In order to do this, I'm going to use the asymptotic expansion of \(M(a,b,z)\) \cite[\href{https://dlmf.nist.gov/13.7.E2}{(13.7.2)}]{NIST:DLMF}:
\begin{align}
	M^{\infty} (a,b,z)&= 
	\frac{\Gamma (b)}{\Gamma (a)} e^z z^{a-b} \sum_{n} \frac{(1-a)_n (b-a)_n}{n!}z^{-n} +\\
	&+ \frac{\Gamma (b)}{\Gamma (b-a)}(-z)^{-a} \sum_{n} \frac{(a)_n(a-b+1)_n}{n!}(-z)^{-n},
\end{align}
where \((c)_n\) is the rising factorial. As \(z\) goes to infinity, all terms in the series vanish, apart from the \(n=0\) ones that yield \(1\). The expansion can therefore be rewritten as
\begin{equation}\label{deriv:Mexpansion}
	M^{\infty} (a,b,z) =
	\frac{\Gamma (b)}{\Gamma (a)} e^z z^{a-b} +
	\frac{\Gamma (b)}{\Gamma (b-a)}(-z)^{-a}
\end{equation}
If we insert our values for \(b\) and \(a\) into \(\Gamma (b-a)\), we get that
\begin{equation}
	\Gamma (b-a) = \Gamma \left( 2- 1 + \frac{i}{2\varepsilon } \right) = \Gamma \left( 1+ \frac{i}{2 \varepsilon } \right) = \Gamma (\overline{a})= \overline{\Gamma (a)}
\end{equation}
We can define \(\eta \) such that \(\Gamma (a) = \vert \Gamma (a) \vert e^{i \eta } \) and write the asymptotic expansion of \(\chi \) as
\begin{align}
	\chi^{\infty}(x) &= Cxe^{i\varepsilon x} \frac{\Gamma (b)}{\vert \Gamma (a) \vert }
	\left( e^{-i \eta } e^z	z^{a-b} + e^{i \eta } (-z)^{-a}  \right) =\\
	&= C x \frac{\Gamma (2) e^{i \eta }}{\Gamma \left(1-\frac{i}{2\varepsilon }\right)}\left( e^{-i \eta } e^{-i\varepsilon x}(-2i\varepsilon x)^{-1-\frac{i}{2\varepsilon }} + e^{i \eta } e^{i \varepsilon x} (2i \varepsilon x)^{-1+ \frac{i}{2\varepsilon }} \right)=\\
	&=C \frac{e^{i \eta - \frac{\pi}{4 \varepsilon }}}{\Gamma \left( 1-\frac{i}{2\varepsilon } \right) \varepsilon } \sin \left( \varepsilon x + \frac{1}{2\varepsilon} \ln (2\varepsilon x) + \eta \right)
\end{align}
By noting that \(\varepsilon x= kr\) and that the logarithm has basically no impact at large \(x\), we conclude that the solution is in the form required by Eq. \eqref{deriv:chi_asymptotic} and that
\begin{equation}
	C = \frac{\Gamma \left( 1- \frac{i}{2\varepsilon }\right) \varepsilon }{e^{i \eta - \frac{\pi}{4 \varepsilon }}}
\end{equation}
We now have the solution we were looking for:
\begin{equation}\label{deriv:solution}
	\chi (x) = \frac{\Gamma \left(1-\frac{i}{2\varepsilon }\right) \varepsilon }{e^{i \eta - \frac{\pi}{4 \varepsilon }}}x e^{i\varepsilon x}M\left(1-\frac{i}{2\varepsilon },2,-2i \varepsilon x\right) 
\end{equation}
When taking the derivative at the origin, the only term that does not vanish is the one where we take the derivative of \(x \), leaving the exponential and the confluent hypergeometric function (which when evaluated at \(x=0\) yield \(1\)):
\begin{align}
	\frac{\mathrm{d}\chi }{\mathrm{d}x} (x=0) &=
	\left.
		\frac{\Gamma \left( 1- \frac{i}{2\varepsilon } \right) \varepsilon }{e^{i \eta -\frac{\pi}{4\varepsilon }}} e^{i\varepsilon x} M\left(1-\frac{i}{2\varepsilon },2,-2 i \varepsilon x\right)
	\right\vert _{x=0} \\
	&= \frac{\Gamma \left( 1- \frac{i}{2\varepsilon }\right) \varepsilon }{e^{i \eta - \frac{\pi}{4\varepsilon }}}
\end{align}
The Sommerfeld enhancement factor is thus
\begin{align}
	S_{\varepsilon } &= \left\vert \Gamma \left(1-\frac{i}{2\varepsilon }\right) e^{-i \eta + \frac{\pi}{4\varepsilon }} \right\vert ^2=\left\vert \Gamma \left(1-\frac{i}{2\varepsilon }\right)\right\vert ^2 e^{\frac{\pi}{2\varepsilon }}=\label{deriv:sef_gamma}\\
	&= \frac{\frac{\pi}{2\varepsilon }}{\sinh \left( \frac{\pi }{2 \varepsilon } \right)}e^{\frac{\pi}{2\varepsilon }} = \frac{\frac{\pi}{\varepsilon }}{1- e^{-\frac{\pi}{\varepsilon }}}=\label{deriv:sef_sinh}\\
	&= 2\pi \frac{\alpha}{v_{rel} } \frac{1}{1- e^{-2\pi \frac{\alpha}{v_{rel} }}},
\end{align}
where I have used Euler's reflection formula \(\Gamma (z) \Gamma (1-z) = \frac{\pi}{\sin(\pi z)}\), the Gamma function recurrence relation \(\Gamma (z+1)=z \Gamma (z)\) and the identity \(\sin (ix)=i \sinh (x)\) to prove the step between Eq. \eqref{deriv:sef_gamma} and Eq. \eqref{deriv:sef_sinh}.