\appendix
\chapter{Reduction to one-body problem}\label{appendix:onebody}
Separating a two-body problem into two separate problems concerning the centre of mass and the relative motion of the two particles is a useful technique that has been applied in this work.

\paragraph{The Schrödinger equation}
If the two particles have the same mass \(m\), the reduced mass is just \(\mu = m /2\) and the transformation can be carried out via the following substitutions:
\begin{equation}\label{eq:twobody_transform}
	\vec{r}\coloneqq \vec{r}_1 - \vec{r}_2 \qquad
	\vec{R} \coloneqq \frac{\vec{r}_1 + \vec{r}_2}{2}
\end{equation}
One can easily show that this leads to the following relations:
\begin{equation}
	\nabla _1 = \frac{1}{2} \nabla _R + \nabla _r
	\qquad
	\nabla _2 = \frac{1}{2} \nabla _R - \nabla _r
\end{equation}
To derive this, I will call \(x^i_1, x^i_2, X^i \) and \(x^i\) the coordinates of \(\vec{r}_1, \vec{r}_2, \vec{R} \) and \(\vec{r}\) respectively. By applying the chain rule,
\begin{equation}
	\frac{\partial }{\partial x^i_1} = \frac{\partial X^i}{\partial x^i_1} \frac{\partial }{\partial X^i} + \frac{\partial x^i}{\partial x^i_1} \frac{\partial }{\partial x^i} = \frac{1}{2} \frac{\partial }{\partial X^i} + \frac{\partial }{\partial x^i},
\end{equation}
and a similar result can be obtained for the second particle. One can verify that the quantum mechanical operators defined through these substitutions still satisfy the required commutation relations. With this result, the typical equation describing two particles in a potential (Eq. \eqref{eq:initial_equation}) turns into a simpler form:
\begin{equation}
	\left[- \frac{1}{8\mu} \nabla _R^{2} - \frac{1}{2\mu } \nabla _r^2 + V(r)\right] \Psi (\vec{R},\vec{r})= E \Psi (\vec{R},\vec{r})
\end{equation}
This equation clearly suggests that the wavefunction can be separated into the product of a function describing the centre of mass and another one describing the system as seen from the centre-of-mass frame: \(\Psi (\vec{R},\vec{r}) = \psi_R \psi _r\). The problem can therefore be studied as two separate problems, and for the sake of this derivation we will only focus on the centre-of-mass description of it. The centre-of-mass Schrödinger equation is
\begin{equation}
	\left[- \frac{1}{2\mu }\nabla^2 + V(r)\right]\psi _r = E_{CM} \psi _r
\end{equation}

\paragraph{Relative velocity Maxwell-Boltzmann distribution}
The relative velocity of the two particles is just the first derivative of the left-side relation in Eq. \eqref{eq:twobody_transform}:
\begin{equation}
	\vec{v}_{rel} = \vec{v}_1 - \vec{v}_2
\end{equation}
As the two particles' velocities are independent of each other, the probability of having a certain \(\vec{v}_{rel} \) is just the product of the two probabilities:
\begin{equation}
	P(\vec{v}_1, \vec{v}_2) = P(\vec{v}_1)P(\vec{v}_2)
\end{equation}
If the velocities are governed by a Maxwell-Boltzmann distribution with velocity dispersion \(\beta \), we find that
\begin{equation}\label{eq:MB_rel_start}
	P(\vec{v}_1, \vec{v}_2) = \left( \frac{1}{2\pi } \right)^3 \frac{1}{\beta ^6} \exp \left( -\frac{v_1^2 + v_2^2}{2\beta ^2} \right)  
\end{equation}
The relations \eqref{eq:twobody_transform} imply that the velocities of the two particles can be expressed in terms of the centre-of-mass velocity and of the relative velocity as follows:
\begin{equation}
	\vec{v}_1 = \vec{v}_{CM} + \frac{1}{2} \vec{v}_{rel}
	\qquad
	\vec{v}_2 = \vec{v}_{CM} - \frac{1}{2} \vec{v}_{rel} 
\end{equation}
This allows us to rewrite Eq. \eqref{eq:MB_rel_start} as
\begin{equation}
	P(\vec{v}_{rel} , \vec{v}_{CM} ) = \left( \frac{1}{2\pi } \right)^3 \frac{1}{\beta ^6} \exp \left( - \frac{2 v_{CM} ^2 + \frac{1}{2}v_{rel} ^2}{2\beta ^2} \right)  
\end{equation}
and to separate it into a Maxwell-Boltzmann distribution for the centre-of-mass velocity and another one for the relative velocity:
\begin{equation}
	P(\vec{v}_{rel} ) = \left( \frac{1}{4\pi } \right)^{\frac{3}{2}} \frac{1}{\beta ^3} \exp \left( - \frac{v_{rel} ^2}{4\beta ^2} \right)  
\end{equation}
Hence, we see that the relative velocity dispersion satisfies \(\beta _{rel}=\sqrt{2} \beta \). Thus, the distribution that describes the modulus of the relative velocity is, as found in \cite{Ferrer_2013},
\begin{equation}
	P(v_{rel} ) = \left( \frac{1}{4\pi } \right) ^{\frac{1}{2}} \frac{v_{rel} ^2}{\beta ^3} \exp \left( - \frac{v_{rel} ^2}{4\beta ^2} \right)  
\end{equation}